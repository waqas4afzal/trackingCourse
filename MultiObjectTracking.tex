\PassOptionsToPackage{dvipsnames}{xcolor}
\documentclass{beamer}
\usetheme{Warsaw}
\usepackage[acronym]{glossaries}
% for two columns
\usepackage{glossary-super}
%\glsdisablehyper %This disables hyper links to glossary items
%%%%%%%%%%%%%%%%%%%%%%%%%%%%%%%%%%%%%%%%%%%%%%
% Syntax for adding new acronym
%
%\newacronym[longplural={plural different from addind s}]{label}{acronym}{Full form}
%
%%%%%%%%%%%%%%%%%%%%%%%%%%%%%%%%%%%%%%%%%%%%%%
%Syntax for adding new symbol
%
% \newglossaryentry{label}{
	% name={name of symbol},
	% description={desription of symbol},
	% symbol={symbol if math use \ensuremath{.}},
	% plural={plural},  %if plural is different from addind s to the end
	% sort={name for sorting alphabetically}
	% }
%
%%%%%%%%%%%%%%%%%%%%%%%%%%%%%%%%%%%%%%%%%%%%%%
%
\newcommand{\sym}[4]{%
	\newglossaryentry{#1}{%
		name={#2},
		description={#3},%symbol={#2},
		sort={#4}
	}%
}%
%Define symbols here
% example definition
%\sym{wsp}{\ensuremath{\mathcal{W}}}{Denotes workspace, the environment in which the agent exists.}{Workspace}
\sym{msmatrix}{\ensuremath{Z}}{Measurement matrix}{msmatrix}
\sym{omatrix}{\ensuremath{X}}{Object state matrix}{omatrix}
\sym{omsmatrix}{\ensuremath{O}}{Object measurement matrix}{omsmatrix}
\sym{cmsmatrix}{\ensuremath{C}}{Clutter measurement matrix}{cmsmatrix}
\sym{msmt}{\ensuremath{\mathbf{z}}}{Measurement vector}{msmt}
\sym{omsmt}{\ensuremath{\mathbf{o}}}{Object measurement vector}{omsmt}
\sym{cmsmt}{\ensuremath{\mathbf{c}}}{Clutter measurement vector}{cmsmt}
\sym{ostate}{\ensuremath{\mathbf{x}}}{Object state vector}{ostate}
\sym{tstep}{\ensuremath{k}}{Time step}{time}
\sym{m}{\ensuremath{m}}{Number of measurements}{Nmsmt}
\sym{dassoc}{\ensuremath{\theta}}{Data association vector}{dassoc}
\sym{setdassoc}{\ensuremath{\Theta}}{Set of all data association vectors}{setdassoc}
\sym{nobj}{\ensuremath{n}}{Number of objects}{nobj}
%Define abbreviations here
% % example
%\newacronym{awgn}{AWGN}{Additive White Gaussian Noise}
%% C
\newacronym{cap}{CAP}{Combat Air Patrol}
\newacronym{cdf}{CDF}{Cumulative Density Function}
%% E
\newacronym{em}{EM}{Electromagnetic}
\newacronym{esm}{ESM}{Electronic Support Measures}
%% G
\newacronym{gnn}{GNN}{Global Nearest Neighbor}
%% H
\newacronym{hmm}{HMM}{Hidden Markov Model}
%% J
\newacronym{jpda}{JPDA}{Joint Probabilistic Data Association}
%% M
\newacronym{mht}{MHT}{Multi Hypothesis Tracker}
\newacronym{mmse}{MMSE}{Minimum Mean Squared Error}
\newacronym{mot}{MOT}{Multiple Object Tracking}
\newacronym{mse}{MSE}{Mean Squared Error}
%% N
\newacronym{nlos}{NLOS}{Non Line-of-Sight}
%% P
\newacronym{pdf}{PDF}{Probability Density Function}
\newacronym{pmf}{PMF}{Probability Mass Function}
%% S
\newacronym{snr}{SNR}{Signal-to-Noise Ratio}
\newacronym{sot}{SOT}{Single Object Tracking}
%% T
\newacronym{tbd}{TBD}{Track-Before-Detect}
%%%%%%%%%%%%%%%%%%%%%%%%%%%%%%%%%%%%%%%%%%%%%%
%%%%%%%%%%%%%%%%%%%%%%%%%%%%%%%%%%%%%%%%%%%%%%
\makeglossaries

\def\mse{\operatorname{MSE}}
\def\mean{\operatorname{E}}
\def\var{\operatorname{Var}}
\usepackage{cleveref}
\title{Multiple Object Tracking: Fundamentals}
\author{Dr. Waqas Afzal}
\begin{document}
\begin{frame}[plain]
    \maketitle
\end{frame}
\begin{frame}{Scope}
	\begin{itemize}
		\item An introduction to \gls{mot} using multiple sensors.
		\item In this part fundamentals of tracking will be covered.
	\end{itemize}
\end{frame}
\begin{frame}{Introduction}
	\begin{itemize}
		\item Multiple sensors-multiple target tracking scenario.
		\item Associating tracks from multiple sensors to a target.
		\item Errors can affect operators ability to analyze formations and consequently, the intent.
	\end{itemize}
\end{frame}
\begin{frame}{Introduction}{Challenges in \gls{mot}}	
	\begin{itemize}
		\item The first challenge is detecting objects. Since the number of dynamic objects is unknown, detection needs to be done at each time step.
		\item The states of the detected objects are also unknown, thus their states need to be estimated at each time step.
		\item New objects can enter the sensor’s range and objects detected in the previous time step can leave the sensor’s range.
		\item Objects can occlude one another leading to missed detections.
		\item False alarms or missed detections due to inherent sensor noise.
		\item Data association
		\begin{itemize}
			\item which detections are from objects detected previously.
			\item which detections are from new objects in the sensors range.
			\item false detections (clutter).
		\end{itemize}
	\end{itemize}
\end{frame}
\begin{frame}{Introduction}{\acrfull{sot} and \acrfull{mot}}
	\begin{itemize}
		\item \textbf{\acrfull{sot}}: \acrshort{sot} is a
		filtering problem. It can be described as the sequential processing of noisy sensor measurements to determine an object’s state:
		\begin{itemize}
			\item Position.
			\item Other properties like kinematics, etc.
			\item Other attributes related to the detected object like \gls{esm} data.
		\end{itemize}
		\item \textbf{\acrfull{mot}}: \gls{mot} is the sequential processing of noisy sensor measurements to determine
		\begin{itemize}
			\item the number of dynamic objects,
			\item and each dynamic object’s state.
		\end{itemize}
		Simply put, \gls{mot}, jointly detects object and applies \gls{sot} to each of these objects.
	\end{itemize}
\end{frame}
\begin{frame}{Introduction}{Types of Tracking}
	\begin{enumerate}
		\item \textbf{Point Object Tracking:} a single detection is returned per object per time step.
		\item \textbf{Extended Object Tracking:} multiple detections are returned per object per time step.
		\item \textbf{Group Object Tracking:} multiple resolvable objects are treated as a group per time step.
		\item \textbf{Tracking with Muli-Path Propagation: } tracking \gls{nlos} dynamic objects.
		\item \textbf{Tracking with Unresolved Measurements:} multiple objects return a single measurement.
	\end{enumerate}
\end{frame}
\section{Bayesian Filtering}
\begin{frame}{Bayesian Filtering}{Introduction}
	\begin{itemize}
		\item Tracking is basically detection of dynamic objects and estimation of their hidden states in the presence of uncertainty.
		\item Bayesian filters have been shown to perform well in estimating hidden states in the presence of uncertainty.
		\item All Bayesian filters successively follow a two step process.
		\begin{itemize}
			\item \textbf{Motion Update (Prediction)}: Use a motion model to predict the state of the objects in the next time-step.
			\item  \textbf{Measurement Update}: Update the object state’s distributions using a measurement model based on the measurement obtained in the current time-
			step.
		\end{itemize}
	\end{itemize}
\end{frame}
\begin{frame}{Bayesian Filtering}{Nomenclature}
	\begin{itemize}
		\item $\gls{ostate}_k$ is the state vector of an object at time $k$.
		\item $\gls{msmt}_k$ is the measurement vector at time $k$ and $\gls{msmt}_{1:\tau}=(\gls{msmt}_1, \gls{msmt}_2, \dots, \gls{msmt}_\tau)$ is a sequence of measurements upto time $\tau$.
		\item \textbf{Bayesian Filtering Recursion}
		\begin{itemize}
			\item \textbf{Prediction (Chapman-Kolmongorov Equation)}
			\begin{equation}
				\label{eq:prior}
				{\color{OliveGreen}p(\gls{ostate}_k|\gls{msmt}_{1:k-1})} = \displaystyle\int p(\gls{ostate}_k|\gls{ostate}_{k-1}){\color{BrickRed}p(\gls{ostate}_{k-1}|\gls{msmt}_{1:k-1})} d\gls{ostate}_{k-1}
			\end{equation}
			\item \textbf{Update (Bayes Rule)}
			\begin{equation}
				\label{eq:posterior}
				{\color{BrickRed}p(\gls{ostate}_k|\gls{msmt}_{1:k})} = 
				\dfrac{p(\gls{msmt}_k|\gls{ostate}_k){\color{OliveGreen}p(\gls{ostate}_k|\gls{msmt}_{1:k-1})}}{\color{NavyBlue}p(\gls{msmt}_k|\gls{msmt}_{1:k-1})}
			\end{equation}
			\item \textbf{Predicted Likelihood (Distribution/Probability of Evidence)}
			\begin{equation}
				{\color{NavyBlue}p(\gls{msmt}_k|\gls{msmt}_{1:k-1})} = \displaystyle\int p(\gls{msmt}_k|\gls{ostate}_k)p(\gls{ostate}_k|\gls{msmt}_{1:k-1}) d\gls{ostate}_k
			\end{equation}
		\end{itemize}
	\end{itemize}
\end{frame}
\begin{frame}{Bayesian Filtering}{Nomenclature}
	\textbf{Bayes Update}:
	\begin{equation}
		\label{eq:posterior}
		{\color{BrickRed}p(\gls{ostate}_k|\gls{msmt}_{1:k})} = 
		\dfrac{p(\gls{msmt}_k|\gls{ostate}_k){\color{OliveGreen}p(\gls{ostate}_k|\gls{msmt}_{1:k-1})}}{\color{NavyBlue}p(\gls{msmt}_k|\gls{msmt}_{1:k-1})}\nonumber
	\end{equation}
	\begin{align}
		{\color{BrickRed}p(\gls{ostate}_k|\gls{msmt}_{1:k})} &\to \text{Posterior.}\nonumber\\
		p(\gls{msmt}_k|\gls{ostate}_k) &\to \text{Measurement Likelihood.}\nonumber\\
		{\color{OliveGreen}p(\gls{ostate}_k|\gls{msmt}_{1:k-1})} &\to \text{Prior.}\nonumber\\
		{\color{NavyBlue}p(\gls{msmt}_k|\gls{msmt}_{1:k-1})} &\to \text{Predicted Likelihood.}\nonumber
	\end{align}
\end{frame}
\begin{frame}{Bayesian Filtering}{Nomenclature}
	Bayesian filtering alternatively applies the prediction and measurement updates to update the posterior of the object state at each time step.
	\begin{align}
		\text{Initial: }&p(\gls{ostate}_1)\nonumber\\
		\text{Update: }&p(\gls{ostate}_1 | \gls{msmt}_1)\nonumber\\
		\text{Prediction: }&p(\gls{ostate}_2 | \gls{msmt}_1)\nonumber\\
		\text{Update: }&p(\gls{ostate}_2 | \gls{msmt}_{1:2})\nonumber\\
		\text{Prediction: }&p(\gls{ostate}_3 | \gls{msmt}_{1:2})\nonumber\\
		\text{Update: }&p(\gls{ostate}_3 | \gls{msmt}_{1:3})\nonumber\\
		\text{Prediction: }&p(\gls{ostate}_4 | \gls{msmt}_{1:3})\nonumber\\
		\text{Update: }&p(\gls{ostate}_4 | \gls{msmt}_{1:4})\nonumber\\
		\text{Prediction: }&p(\gls{ostate}_5 | \gls{msmt}_{1:4})\nonumber
	\end{align}
\end{frame}
\end{document}
